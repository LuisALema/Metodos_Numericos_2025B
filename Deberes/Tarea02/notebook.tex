\documentclass[11pt]{article}

    \usepackage[breakable]{tcolorbox}
    \usepackage{parskip} % Stop auto-indenting (to mimic markdown behaviour)
    

    % Basic figure setup, for now with no caption control since it's done
    % automatically by Pandoc (which extracts ![](path) syntax from Markdown).
    \usepackage{graphicx}
    % Keep aspect ratio if custom image width or height is specified
    \setkeys{Gin}{keepaspectratio}
    % Maintain compatibility with old templates. Remove in nbconvert 6.0
    \let\Oldincludegraphics\includegraphics
    % Ensure that by default, figures have no caption (until we provide a
    % proper Figure object with a Caption API and a way to capture that
    % in the conversion process - todo).
    \usepackage{caption}
    \DeclareCaptionFormat{nocaption}{}
    \captionsetup{format=nocaption,aboveskip=0pt,belowskip=0pt}

    \usepackage{float}
    \floatplacement{figure}{H} % forces figures to be placed at the correct location
    \usepackage{xcolor} % Allow colors to be defined
    \usepackage{enumerate} % Needed for markdown enumerations to work
    \usepackage{geometry} % Used to adjust the document margins
    \usepackage{amsmath} % Equations
    \usepackage{amssymb} % Equations
    \usepackage{textcomp} % defines textquotesingle
    % Hack from http://tex.stackexchange.com/a/47451/13684:
    \AtBeginDocument{%
        \def\PYZsq{\textquotesingle}% Upright quotes in Pygmentized code
    }
    \usepackage{upquote} % Upright quotes for verbatim code
    \usepackage{eurosym} % defines \euro

    \usepackage{iftex}
    \ifPDFTeX
        \usepackage[T1]{fontenc}
        \IfFileExists{alphabeta.sty}{
              \usepackage{alphabeta}
          }{
              \usepackage[mathletters]{ucs}
              \usepackage[utf8x]{inputenc}
          }
    \else
        \usepackage{fontspec}
        \usepackage{unicode-math}
    \fi

    \usepackage{fancyvrb} % verbatim replacement that allows latex
    \usepackage{grffile} % extends the file name processing of package graphics
                         % to support a larger range
    \makeatletter % fix for old versions of grffile with XeLaTeX
    \@ifpackagelater{grffile}{2019/11/01}
    {
      % Do nothing on new versions
    }
    {
      \def\Gread@@xetex#1{%
        \IfFileExists{"\Gin@base".bb}%
        {\Gread@eps{\Gin@base.bb}}%
        {\Gread@@xetex@aux#1}%
      }
    }
    \makeatother
    \usepackage[Export]{adjustbox} % Used to constrain images to a maximum size
    \adjustboxset{max size={0.9\linewidth}{0.9\paperheight}}

    % The hyperref package gives us a pdf with properly built
    % internal navigation ('pdf bookmarks' for the table of contents,
    % internal cross-reference links, web links for URLs, etc.)
    \usepackage{hyperref}
    % The default LaTeX title has an obnoxious amount of whitespace. By default,
    % titling removes some of it. It also provides customization options.
    \usepackage{titling}
    \usepackage{longtable} % longtable support required by pandoc >1.10
    \usepackage{booktabs}  % table support for pandoc > 1.12.2
    \usepackage{array}     % table support for pandoc >= 2.11.3
    \usepackage{calc}      % table minipage width calculation for pandoc >= 2.11.1
    \usepackage[inline]{enumitem} % IRkernel/repr support (it uses the enumerate* environment)
    \usepackage[normalem]{ulem} % ulem is needed to support strikethroughs (\sout)
                                % normalem makes italics be italics, not underlines
    \usepackage{soul}      % strikethrough (\st) support for pandoc >= 3.0.0
    \usepackage{mathrsfs}
    

    
    % Colors for the hyperref package
    \definecolor{urlcolor}{rgb}{0,.145,.698}
    \definecolor{linkcolor}{rgb}{.71,0.21,0.01}
    \definecolor{citecolor}{rgb}{.12,.54,.11}

    % ANSI colors
    \definecolor{ansi-black}{HTML}{3E424D}
    \definecolor{ansi-black-intense}{HTML}{282C36}
    \definecolor{ansi-red}{HTML}{E75C58}
    \definecolor{ansi-red-intense}{HTML}{B22B31}
    \definecolor{ansi-green}{HTML}{00A250}
    \definecolor{ansi-green-intense}{HTML}{007427}
    \definecolor{ansi-yellow}{HTML}{DDB62B}
    \definecolor{ansi-yellow-intense}{HTML}{B27D12}
    \definecolor{ansi-blue}{HTML}{208FFB}
    \definecolor{ansi-blue-intense}{HTML}{0065CA}
    \definecolor{ansi-magenta}{HTML}{D160C4}
    \definecolor{ansi-magenta-intense}{HTML}{A03196}
    \definecolor{ansi-cyan}{HTML}{60C6C8}
    \definecolor{ansi-cyan-intense}{HTML}{258F8F}
    \definecolor{ansi-white}{HTML}{C5C1B4}
    \definecolor{ansi-white-intense}{HTML}{A1A6B2}
    \definecolor{ansi-default-inverse-fg}{HTML}{FFFFFF}
    \definecolor{ansi-default-inverse-bg}{HTML}{000000}

    % common color for the border for error outputs.
    \definecolor{outerrorbackground}{HTML}{FFDFDF}

    % commands and environments needed by pandoc snippets
    % extracted from the output of `pandoc -s`
    \providecommand{\tightlist}{%
      \setlength{\itemsep}{0pt}\setlength{\parskip}{0pt}}
    \DefineVerbatimEnvironment{Highlighting}{Verbatim}{commandchars=\\\{\}}
    % Add ',fontsize=\small' for more characters per line
    \newenvironment{Shaded}{}{}
    \newcommand{\KeywordTok}[1]{\textcolor[rgb]{0.00,0.44,0.13}{\textbf{{#1}}}}
    \newcommand{\DataTypeTok}[1]{\textcolor[rgb]{0.56,0.13,0.00}{{#1}}}
    \newcommand{\DecValTok}[1]{\textcolor[rgb]{0.25,0.63,0.44}{{#1}}}
    \newcommand{\BaseNTok}[1]{\textcolor[rgb]{0.25,0.63,0.44}{{#1}}}
    \newcommand{\FloatTok}[1]{\textcolor[rgb]{0.25,0.63,0.44}{{#1}}}
    \newcommand{\CharTok}[1]{\textcolor[rgb]{0.25,0.44,0.63}{{#1}}}
    \newcommand{\StringTok}[1]{\textcolor[rgb]{0.25,0.44,0.63}{{#1}}}
    \newcommand{\CommentTok}[1]{\textcolor[rgb]{0.38,0.63,0.69}{\textit{{#1}}}}
    \newcommand{\OtherTok}[1]{\textcolor[rgb]{0.00,0.44,0.13}{{#1}}}
    \newcommand{\AlertTok}[1]{\textcolor[rgb]{1.00,0.00,0.00}{\textbf{{#1}}}}
    \newcommand{\FunctionTok}[1]{\textcolor[rgb]{0.02,0.16,0.49}{{#1}}}
    \newcommand{\RegionMarkerTok}[1]{{#1}}
    \newcommand{\ErrorTok}[1]{\textcolor[rgb]{1.00,0.00,0.00}{\textbf{{#1}}}}
    \newcommand{\NormalTok}[1]{{#1}}

    % Additional commands for more recent versions of Pandoc
    \newcommand{\ConstantTok}[1]{\textcolor[rgb]{0.53,0.00,0.00}{{#1}}}
    \newcommand{\SpecialCharTok}[1]{\textcolor[rgb]{0.25,0.44,0.63}{{#1}}}
    \newcommand{\VerbatimStringTok}[1]{\textcolor[rgb]{0.25,0.44,0.63}{{#1}}}
    \newcommand{\SpecialStringTok}[1]{\textcolor[rgb]{0.73,0.40,0.53}{{#1}}}
    \newcommand{\ImportTok}[1]{{#1}}
    \newcommand{\DocumentationTok}[1]{\textcolor[rgb]{0.73,0.13,0.13}{\textit{{#1}}}}
    \newcommand{\AnnotationTok}[1]{\textcolor[rgb]{0.38,0.63,0.69}{\textbf{\textit{{#1}}}}}
    \newcommand{\CommentVarTok}[1]{\textcolor[rgb]{0.38,0.63,0.69}{\textbf{\textit{{#1}}}}}
    \newcommand{\VariableTok}[1]{\textcolor[rgb]{0.10,0.09,0.49}{{#1}}}
    \newcommand{\ControlFlowTok}[1]{\textcolor[rgb]{0.00,0.44,0.13}{\textbf{{#1}}}}
    \newcommand{\OperatorTok}[1]{\textcolor[rgb]{0.40,0.40,0.40}{{#1}}}
    \newcommand{\BuiltInTok}[1]{{#1}}
    \newcommand{\ExtensionTok}[1]{{#1}}
    \newcommand{\PreprocessorTok}[1]{\textcolor[rgb]{0.74,0.48,0.00}{{#1}}}
    \newcommand{\AttributeTok}[1]{\textcolor[rgb]{0.49,0.56,0.16}{{#1}}}
    \newcommand{\InformationTok}[1]{\textcolor[rgb]{0.38,0.63,0.69}{\textbf{\textit{{#1}}}}}
    \newcommand{\WarningTok}[1]{\textcolor[rgb]{0.38,0.63,0.69}{\textbf{\textit{{#1}}}}}
    \makeatletter
    \newsavebox\pandoc@box
    \newcommand*\pandocbounded[1]{%
      \sbox\pandoc@box{#1}%
      % scaling factors for width and height
      \Gscale@div\@tempa\textheight{\dimexpr\ht\pandoc@box+\dp\pandoc@box\relax}%
      \Gscale@div\@tempb\linewidth{\wd\pandoc@box}%
      % select the smaller of both
      \ifdim\@tempb\p@<\@tempa\p@
        \let\@tempa\@tempb
      \fi
      % scaling accordingly (\@tempa < 1)
      \ifdim\@tempa\p@<\p@
        \scalebox{\@tempa}{\usebox\pandoc@box}%
      % scaling not needed, use as it is
      \else
        \usebox{\pandoc@box}%
      \fi
    }
    \makeatother

    % Define a nice break command that doesn't care if a line doesn't already
    % exist.
    \def\br{\hspace*{\fill} \\* }
    % Math Jax compatibility definitions
    \def\gt{>}
    \def\lt{<}
    \let\Oldtex\TeX
    \let\Oldlatex\LaTeX
    \renewcommand{\TeX}{\textrm{\Oldtex}}
    \renewcommand{\LaTeX}{\textrm{\Oldlatex}}
    % Document parameters
    % Document title
    \title{Tarea 02 Luis Lema}
    
    
    
    
    
    
    
% Pygments definitions
\makeatletter
\def\PY@reset{\let\PY@it=\relax \let\PY@bf=\relax%
    \let\PY@ul=\relax \let\PY@tc=\relax%
    \let\PY@bc=\relax \let\PY@ff=\relax}
\def\PY@tok#1{\csname PY@tok@#1\endcsname}
\def\PY@toks#1+{\ifx\relax#1\empty\else%
    \PY@tok{#1}\expandafter\PY@toks\fi}
\def\PY@do#1{\PY@bc{\PY@tc{\PY@ul{%
    \PY@it{\PY@bf{\PY@ff{#1}}}}}}}
\def\PY#1#2{\PY@reset\PY@toks#1+\relax+\PY@do{#2}}

\@namedef{PY@tok@w}{\def\PY@tc##1{\textcolor[rgb]{0.73,0.73,0.73}{##1}}}
\@namedef{PY@tok@c}{\let\PY@it=\textit\def\PY@tc##1{\textcolor[rgb]{0.24,0.48,0.48}{##1}}}
\@namedef{PY@tok@cp}{\def\PY@tc##1{\textcolor[rgb]{0.61,0.40,0.00}{##1}}}
\@namedef{PY@tok@k}{\let\PY@bf=\textbf\def\PY@tc##1{\textcolor[rgb]{0.00,0.50,0.00}{##1}}}
\@namedef{PY@tok@kp}{\def\PY@tc##1{\textcolor[rgb]{0.00,0.50,0.00}{##1}}}
\@namedef{PY@tok@kt}{\def\PY@tc##1{\textcolor[rgb]{0.69,0.00,0.25}{##1}}}
\@namedef{PY@tok@o}{\def\PY@tc##1{\textcolor[rgb]{0.40,0.40,0.40}{##1}}}
\@namedef{PY@tok@ow}{\let\PY@bf=\textbf\def\PY@tc##1{\textcolor[rgb]{0.67,0.13,1.00}{##1}}}
\@namedef{PY@tok@nb}{\def\PY@tc##1{\textcolor[rgb]{0.00,0.50,0.00}{##1}}}
\@namedef{PY@tok@nf}{\def\PY@tc##1{\textcolor[rgb]{0.00,0.00,1.00}{##1}}}
\@namedef{PY@tok@nc}{\let\PY@bf=\textbf\def\PY@tc##1{\textcolor[rgb]{0.00,0.00,1.00}{##1}}}
\@namedef{PY@tok@nn}{\let\PY@bf=\textbf\def\PY@tc##1{\textcolor[rgb]{0.00,0.00,1.00}{##1}}}
\@namedef{PY@tok@ne}{\let\PY@bf=\textbf\def\PY@tc##1{\textcolor[rgb]{0.80,0.25,0.22}{##1}}}
\@namedef{PY@tok@nv}{\def\PY@tc##1{\textcolor[rgb]{0.10,0.09,0.49}{##1}}}
\@namedef{PY@tok@no}{\def\PY@tc##1{\textcolor[rgb]{0.53,0.00,0.00}{##1}}}
\@namedef{PY@tok@nl}{\def\PY@tc##1{\textcolor[rgb]{0.46,0.46,0.00}{##1}}}
\@namedef{PY@tok@ni}{\let\PY@bf=\textbf\def\PY@tc##1{\textcolor[rgb]{0.44,0.44,0.44}{##1}}}
\@namedef{PY@tok@na}{\def\PY@tc##1{\textcolor[rgb]{0.41,0.47,0.13}{##1}}}
\@namedef{PY@tok@nt}{\let\PY@bf=\textbf\def\PY@tc##1{\textcolor[rgb]{0.00,0.50,0.00}{##1}}}
\@namedef{PY@tok@nd}{\def\PY@tc##1{\textcolor[rgb]{0.67,0.13,1.00}{##1}}}
\@namedef{PY@tok@s}{\def\PY@tc##1{\textcolor[rgb]{0.73,0.13,0.13}{##1}}}
\@namedef{PY@tok@sd}{\let\PY@it=\textit\def\PY@tc##1{\textcolor[rgb]{0.73,0.13,0.13}{##1}}}
\@namedef{PY@tok@si}{\let\PY@bf=\textbf\def\PY@tc##1{\textcolor[rgb]{0.64,0.35,0.47}{##1}}}
\@namedef{PY@tok@se}{\let\PY@bf=\textbf\def\PY@tc##1{\textcolor[rgb]{0.67,0.36,0.12}{##1}}}
\@namedef{PY@tok@sr}{\def\PY@tc##1{\textcolor[rgb]{0.64,0.35,0.47}{##1}}}
\@namedef{PY@tok@ss}{\def\PY@tc##1{\textcolor[rgb]{0.10,0.09,0.49}{##1}}}
\@namedef{PY@tok@sx}{\def\PY@tc##1{\textcolor[rgb]{0.00,0.50,0.00}{##1}}}
\@namedef{PY@tok@m}{\def\PY@tc##1{\textcolor[rgb]{0.40,0.40,0.40}{##1}}}
\@namedef{PY@tok@gh}{\let\PY@bf=\textbf\def\PY@tc##1{\textcolor[rgb]{0.00,0.00,0.50}{##1}}}
\@namedef{PY@tok@gu}{\let\PY@bf=\textbf\def\PY@tc##1{\textcolor[rgb]{0.50,0.00,0.50}{##1}}}
\@namedef{PY@tok@gd}{\def\PY@tc##1{\textcolor[rgb]{0.63,0.00,0.00}{##1}}}
\@namedef{PY@tok@gi}{\def\PY@tc##1{\textcolor[rgb]{0.00,0.52,0.00}{##1}}}
\@namedef{PY@tok@gr}{\def\PY@tc##1{\textcolor[rgb]{0.89,0.00,0.00}{##1}}}
\@namedef{PY@tok@ge}{\let\PY@it=\textit}
\@namedef{PY@tok@gs}{\let\PY@bf=\textbf}
\@namedef{PY@tok@ges}{\let\PY@bf=\textbf\let\PY@it=\textit}
\@namedef{PY@tok@gp}{\let\PY@bf=\textbf\def\PY@tc##1{\textcolor[rgb]{0.00,0.00,0.50}{##1}}}
\@namedef{PY@tok@go}{\def\PY@tc##1{\textcolor[rgb]{0.44,0.44,0.44}{##1}}}
\@namedef{PY@tok@gt}{\def\PY@tc##1{\textcolor[rgb]{0.00,0.27,0.87}{##1}}}
\@namedef{PY@tok@err}{\def\PY@bc##1{{\setlength{\fboxsep}{\string -\fboxrule}\fcolorbox[rgb]{1.00,0.00,0.00}{1,1,1}{\strut ##1}}}}
\@namedef{PY@tok@kc}{\let\PY@bf=\textbf\def\PY@tc##1{\textcolor[rgb]{0.00,0.50,0.00}{##1}}}
\@namedef{PY@tok@kd}{\let\PY@bf=\textbf\def\PY@tc##1{\textcolor[rgb]{0.00,0.50,0.00}{##1}}}
\@namedef{PY@tok@kn}{\let\PY@bf=\textbf\def\PY@tc##1{\textcolor[rgb]{0.00,0.50,0.00}{##1}}}
\@namedef{PY@tok@kr}{\let\PY@bf=\textbf\def\PY@tc##1{\textcolor[rgb]{0.00,0.50,0.00}{##1}}}
\@namedef{PY@tok@bp}{\def\PY@tc##1{\textcolor[rgb]{0.00,0.50,0.00}{##1}}}
\@namedef{PY@tok@fm}{\def\PY@tc##1{\textcolor[rgb]{0.00,0.00,1.00}{##1}}}
\@namedef{PY@tok@vc}{\def\PY@tc##1{\textcolor[rgb]{0.10,0.09,0.49}{##1}}}
\@namedef{PY@tok@vg}{\def\PY@tc##1{\textcolor[rgb]{0.10,0.09,0.49}{##1}}}
\@namedef{PY@tok@vi}{\def\PY@tc##1{\textcolor[rgb]{0.10,0.09,0.49}{##1}}}
\@namedef{PY@tok@vm}{\def\PY@tc##1{\textcolor[rgb]{0.10,0.09,0.49}{##1}}}
\@namedef{PY@tok@sa}{\def\PY@tc##1{\textcolor[rgb]{0.73,0.13,0.13}{##1}}}
\@namedef{PY@tok@sb}{\def\PY@tc##1{\textcolor[rgb]{0.73,0.13,0.13}{##1}}}
\@namedef{PY@tok@sc}{\def\PY@tc##1{\textcolor[rgb]{0.73,0.13,0.13}{##1}}}
\@namedef{PY@tok@dl}{\def\PY@tc##1{\textcolor[rgb]{0.73,0.13,0.13}{##1}}}
\@namedef{PY@tok@s2}{\def\PY@tc##1{\textcolor[rgb]{0.73,0.13,0.13}{##1}}}
\@namedef{PY@tok@sh}{\def\PY@tc##1{\textcolor[rgb]{0.73,0.13,0.13}{##1}}}
\@namedef{PY@tok@s1}{\def\PY@tc##1{\textcolor[rgb]{0.73,0.13,0.13}{##1}}}
\@namedef{PY@tok@mb}{\def\PY@tc##1{\textcolor[rgb]{0.40,0.40,0.40}{##1}}}
\@namedef{PY@tok@mf}{\def\PY@tc##1{\textcolor[rgb]{0.40,0.40,0.40}{##1}}}
\@namedef{PY@tok@mh}{\def\PY@tc##1{\textcolor[rgb]{0.40,0.40,0.40}{##1}}}
\@namedef{PY@tok@mi}{\def\PY@tc##1{\textcolor[rgb]{0.40,0.40,0.40}{##1}}}
\@namedef{PY@tok@il}{\def\PY@tc##1{\textcolor[rgb]{0.40,0.40,0.40}{##1}}}
\@namedef{PY@tok@mo}{\def\PY@tc##1{\textcolor[rgb]{0.40,0.40,0.40}{##1}}}
\@namedef{PY@tok@ch}{\let\PY@it=\textit\def\PY@tc##1{\textcolor[rgb]{0.24,0.48,0.48}{##1}}}
\@namedef{PY@tok@cm}{\let\PY@it=\textit\def\PY@tc##1{\textcolor[rgb]{0.24,0.48,0.48}{##1}}}
\@namedef{PY@tok@cpf}{\let\PY@it=\textit\def\PY@tc##1{\textcolor[rgb]{0.24,0.48,0.48}{##1}}}
\@namedef{PY@tok@c1}{\let\PY@it=\textit\def\PY@tc##1{\textcolor[rgb]{0.24,0.48,0.48}{##1}}}
\@namedef{PY@tok@cs}{\let\PY@it=\textit\def\PY@tc##1{\textcolor[rgb]{0.24,0.48,0.48}{##1}}}

\def\PYZbs{\char`\\}
\def\PYZus{\char`\_}
\def\PYZob{\char`\{}
\def\PYZcb{\char`\}}
\def\PYZca{\char`\^}
\def\PYZam{\char`\&}
\def\PYZlt{\char`\<}
\def\PYZgt{\char`\>}
\def\PYZsh{\char`\#}
\def\PYZpc{\char`\%}
\def\PYZdl{\char`\$}
\def\PYZhy{\char`\-}
\def\PYZsq{\char`\'}
\def\PYZdq{\char`\"}
\def\PYZti{\char`\~}
% for compatibility with earlier versions
\def\PYZat{@}
\def\PYZlb{[}
\def\PYZrb{]}
\makeatother


    % For linebreaks inside Verbatim environment from package fancyvrb.
    \makeatletter
        \newbox\Wrappedcontinuationbox
        \newbox\Wrappedvisiblespacebox
        \newcommand*\Wrappedvisiblespace {\textcolor{red}{\textvisiblespace}}
        \newcommand*\Wrappedcontinuationsymbol {\textcolor{red}{\llap{\tiny$\m@th\hookrightarrow$}}}
        \newcommand*\Wrappedcontinuationindent {3ex }
        \newcommand*\Wrappedafterbreak {\kern\Wrappedcontinuationindent\copy\Wrappedcontinuationbox}
        % Take advantage of the already applied Pygments mark-up to insert
        % potential linebreaks for TeX processing.
        %        {, <, #, %, $, ' and ": go to next line.
        %        _, }, ^, &, >, - and ~: stay at end of broken line.
        % Use of \textquotesingle for straight quote.
        \newcommand*\Wrappedbreaksatspecials {%
            \def\PYGZus{\discretionary{\char`\_}{\Wrappedafterbreak}{\char`\_}}%
            \def\PYGZob{\discretionary{}{\Wrappedafterbreak\char`\{}{\char`\{}}%
            \def\PYGZcb{\discretionary{\char`\}}{\Wrappedafterbreak}{\char`\}}}%
            \def\PYGZca{\discretionary{\char`\^}{\Wrappedafterbreak}{\char`\^}}%
            \def\PYGZam{\discretionary{\char`\&}{\Wrappedafterbreak}{\char`\&}}%
            \def\PYGZlt{\discretionary{}{\Wrappedafterbreak\char`\<}{\char`\<}}%
            \def\PYGZgt{\discretionary{\char`\>}{\Wrappedafterbreak}{\char`\>}}%
            \def\PYGZsh{\discretionary{}{\Wrappedafterbreak\char`\#}{\char`\#}}%
            \def\PYGZpc{\discretionary{}{\Wrappedafterbreak\char`\%}{\char`\%}}%
            \def\PYGZdl{\discretionary{}{\Wrappedafterbreak\char`\$}{\char`\$}}%
            \def\PYGZhy{\discretionary{\char`\-}{\Wrappedafterbreak}{\char`\-}}%
            \def\PYGZsq{\discretionary{}{\Wrappedafterbreak\textquotesingle}{\textquotesingle}}%
            \def\PYGZdq{\discretionary{}{\Wrappedafterbreak\char`\"}{\char`\"}}%
            \def\PYGZti{\discretionary{\char`\~}{\Wrappedafterbreak}{\char`\~}}%
        }
        % Some characters . , ; ? ! / are not pygmentized.
        % This macro makes them "active" and they will insert potential linebreaks
        \newcommand*\Wrappedbreaksatpunct {%
            \lccode`\~`\.\lowercase{\def~}{\discretionary{\hbox{\char`\.}}{\Wrappedafterbreak}{\hbox{\char`\.}}}%
            \lccode`\~`\,\lowercase{\def~}{\discretionary{\hbox{\char`\,}}{\Wrappedafterbreak}{\hbox{\char`\,}}}%
            \lccode`\~`\;\lowercase{\def~}{\discretionary{\hbox{\char`\;}}{\Wrappedafterbreak}{\hbox{\char`\;}}}%
            \lccode`\~`\:\lowercase{\def~}{\discretionary{\hbox{\char`\:}}{\Wrappedafterbreak}{\hbox{\char`\:}}}%
            \lccode`\~`\?\lowercase{\def~}{\discretionary{\hbox{\char`\?}}{\Wrappedafterbreak}{\hbox{\char`\?}}}%
            \lccode`\~`\!\lowercase{\def~}{\discretionary{\hbox{\char`\!}}{\Wrappedafterbreak}{\hbox{\char`\!}}}%
            \lccode`\~`\/\lowercase{\def~}{\discretionary{\hbox{\char`\/}}{\Wrappedafterbreak}{\hbox{\char`\/}}}%
            \catcode`\.\active
            \catcode`\,\active
            \catcode`\;\active
            \catcode`\:\active
            \catcode`\?\active
            \catcode`\!\active
            \catcode`\/\active
            \lccode`\~`\~
        }
    \makeatother

    \let\OriginalVerbatim=\Verbatim
    \makeatletter
    \renewcommand{\Verbatim}[1][1]{%
        %\parskip\z@skip
        \sbox\Wrappedcontinuationbox {\Wrappedcontinuationsymbol}%
        \sbox\Wrappedvisiblespacebox {\FV@SetupFont\Wrappedvisiblespace}%
        \def\FancyVerbFormatLine ##1{\hsize\linewidth
            \vtop{\raggedright\hyphenpenalty\z@\exhyphenpenalty\z@
                \doublehyphendemerits\z@\finalhyphendemerits\z@
                \strut ##1\strut}%
        }%
        % If the linebreak is at a space, the latter will be displayed as visible
        % space at end of first line, and a continuation symbol starts next line.
        % Stretch/shrink are however usually zero for typewriter font.
        \def\FV@Space {%
            \nobreak\hskip\z@ plus\fontdimen3\font minus\fontdimen4\font
            \discretionary{\copy\Wrappedvisiblespacebox}{\Wrappedafterbreak}
            {\kern\fontdimen2\font}%
        }%

        % Allow breaks at special characters using \PYG... macros.
        \Wrappedbreaksatspecials
        % Breaks at punctuation characters . , ; ? ! and / need catcode=\active
        \OriginalVerbatim[#1,codes*=\Wrappedbreaksatpunct]%
    }
    \makeatother

    % Exact colors from NB
    \definecolor{incolor}{HTML}{303F9F}
    \definecolor{outcolor}{HTML}{D84315}
    \definecolor{cellborder}{HTML}{CFCFCF}
    \definecolor{cellbackground}{HTML}{F7F7F7}

    % prompt
    \makeatletter
    \newcommand{\boxspacing}{\kern\kvtcb@left@rule\kern\kvtcb@boxsep}
    \makeatother
    \newcommand{\prompt}[4]{
        {\ttfamily\llap{{\color{#2}[#3]:\hspace{3pt}#4}}\vspace{-\baselineskip}}
    }
    

    
    % Prevent overflowing lines due to hard-to-break entities
    \sloppy
    % Setup hyperref package
    \hypersetup{
      breaklinks=true,  % so long urls are correctly broken across lines
      colorlinks=true,
      urlcolor=urlcolor,
      linkcolor=linkcolor,
      citecolor=citecolor,
      }
    % Slightly bigger margins than the latex defaults
    
    \geometry{verbose,tmargin=1in,bmargin=1in,lmargin=1in,rmargin=1in}
    
    

\begin{document}
    
    \maketitle
    
    

    
    \section{ESCUELA POLITÉCNICA
NACIONAL}\label{escuela-polituxe9cnica-nacional}

\subsection{FACULTAD DE INGENIERÍA DE
SISTEMAS}\label{facultad-de-ingenieruxeda-de-sistemas}

\subsubsection{MÉTODOS NUMÉRICOS}\label{muxe9todos-numuxe9ricos}

\paragraph{INGENIERÍA DE SISTEMAS INFORMÁTICOS Y DE
COMPUTACIÓN}\label{ingenieruxeda-de-sistemas-informuxe1ticos-y-de-computaciuxf3n}

\textbf{Nombre:} Luis Alexander Lema Delgado \textbf{Curso:} GR1CC\\
\textbf{Fecha:} 04/11/2025

\begin{center}\rule{0.5\linewidth}{0.5pt}\end{center}

\subsection{{[}Tarea 02{]} Ejercicios Unidad
01-A}\label{tarea-02-ejercicios-unidad-01-a}

\subsubsection{CONJUNTO DE EJERCICIOS 1}\label{conjunto-de-ejercicios-1}

Resuelva los siguientes ejercicios, tome en cuenta que debe mostrar el
desarrollo completo del ejercicio.

\begin{center}\rule{0.5\linewidth}{0.5pt}\end{center}

\subsubsection{\texorpdfstring{1. Calcule los errores absoluto y
relativo en las aproximaciones de \(p\) por
\(p^*\)}{1. Calcule los errores absoluto y relativo en las aproximaciones de p por p\^{}*}}\label{calcule-los-errores-absoluto-y-relativo-en-las-aproximaciones-de-p-por-p}

\paragraph{\texorpdfstring{a. \(p = \pi\),
\(p^* = 22/7\)}{a. p = \textbackslash pi, p\^{}* = 22/7}}\label{a.-p-pi-p-227}

\textbf{Error absoluto} = \(|p - p^*|\)\\
\(= \left| \pi - \frac{22}{7} \right|\)\\
\(= 0.0012644892673496777\)

\textbf{Error Relativo} = \(\frac{|p - p^*|}{|p|}\)\\
\(= \frac{\left| \pi - \frac{22}{7} \right|}{\left| \pi \right|}\)\\
\(= 0.0004024994347707008\)\\
\(= 4.025 \times 10^{-4}\)

\paragraph{\texorpdfstring{b. \(p = \pi\),
\(p^* = 3.1416\)}{b. p = \textbackslash pi, p\^{}* = 3.1416}}\label{b.-p-pi-p-3.1416}

\textbf{Error absoluto} = \(|p - p^*|\)\\
\(= \left| \pi - 3.1416 \right|\)\\
\(= 7.346410206832132 \times 10^{-6}\)

\textbf{Error Relativo} = \(\frac{|p - p^*|}{|p|}\)\\
\(= \frac{|\pi - 3.1416|}{|\pi|}\)\\
\(= 2.3384349967961744 \times 10^{-6}\)

\paragraph{\texorpdfstring{c.~\(p = e\),
\(p^* = 2.718\)}{c.~p = e, p\^{}* = 2.718}}\label{c.-p-e-p-2.718}

\textbf{Error absoluto} = \(|p - p^*|\)\\
\(= |e - 2.718| = 0.000281828459045119\)

\textbf{Error Relativo} = \(\frac{|p - p^*|}{|p|}\)\\
\(= \frac{|e - 2.718|}{|e|} = 0.00010367889601972718\)

\paragraph{\texorpdfstring{d.~\(p = \sqrt{2}\),
\(p^* = 1.414\)}{d.~p = \textbackslash sqrt\{2\}, p\^{}* = 1.414}}\label{d.-p-sqrt2-p-1.414}

\textbf{Error absoluto} = \(|p - p^*|\)\\
\(= |\sqrt{2} - 1.414| = 0.00021356237309522186\)

\textbf{Error Relativo} = \(\frac{|p - p^*|}{|p|}\)\\
\(= \frac{|\sqrt{2} - 1.414|}{|\sqrt{2}|} = 0.00015101140222192286\)

\begin{center}\rule{0.5\linewidth}{0.5pt}\end{center}

\subsubsection{\texorpdfstring{2. Calcule los errores absoluto y
relativo en las aproximaciones de \(p\) por
\(p^*\)}{2. Calcule los errores absoluto y relativo en las aproximaciones de p por p\^{}*}}\label{calcule-los-errores-absoluto-y-relativo-en-las-aproximaciones-de-p-por-p-1}

\paragraph{\texorpdfstring{a. \(p = e^{10}\),
\(p^* = 22000\)}{a. p = e\^{}\{10\}, p\^{}* = 22000}}\label{a.-p-e10-p-22000}

\textbf{Error absoluto} = \(|p - p^*|\)\\
\(= |e^{10} - 22000|\)\\
\(= 26.465794806717895\)

\textbf{Error Relativo} = \(\frac{|p - p^*|}{|p|}\)\\
\(= \frac{|e^{10} - 22000|}{|e^{10}|}\)\\
\(= 0.0012015452253333286\)

\paragraph{\texorpdfstring{b. \(p = 10^\pi\),
\(p^* = 1400\)}{b. p = 10\^{}\textbackslash pi, p\^{}* = 1400}}\label{b.-p-10pi-p-1400}

\textbf{Error absoluto} = \(|p - p^*|\)\\
\(= |10^\pi - 1400|\)\\
\(= 14.544268632989315\)

\textbf{Error Relativo} = \(\frac{|p - p^*|}{|p|}\)\\
\(= \frac{|10^\pi - 1400|}{|10^\pi|}\)\\
\(= 0.010497822704619136\)

\paragraph{\texorpdfstring{c.~\(p = 8!\),
\(p^* = 39900\)}{c.~p = 8!, p\^{}* = 39900}}\label{c.-p-8-p-39900}

\textbf{Error absoluto} = \(|p - p^*|\)\\
\(= |8! - 39900|\)\\
\(= 420\)

\textbf{Error Relativo} = \(\frac{|p - p^*|}{|p|}\)\\
\(= \frac{|8! - 39900|}{|8!|}\)\\
\(= 0.010416666666666666\)

\paragraph{\texorpdfstring{d.~\(p = 9!\),
\(p^* = \sqrt{18\pi} \left( \frac{9}{e} \right)^9\)}{d.~p = 9!, p\^{}* = \textbackslash sqrt\{18\textbackslash pi\} \textbackslash left( \textbackslash frac\{9\}\{e\} \textbackslash right)\^{}9}}\label{d.-p-9-p-sqrt18pi-left-frac9e-right9}

\textbf{Error absoluto} = \(|p - p^*|\)\\
\(= \left| 9! - \sqrt{18\pi} \left( \frac{9}{e} \right)^9 \right|\)\\
\(= 3343.1271580516477\)

\textbf{Error Relativo} = \(\frac{|p - p^*|}{|p|}\)\\
\(= \frac{\left| 9! - \sqrt{18\pi} \left( \frac{9}{e} \right)^9 \right|}{\left| 9! \right|}\)\\
\(= 0.009212762230080598\)

\begin{center}\rule{0.5\linewidth}{0.5pt}\end{center}

\subsubsection{\texorpdfstring{3. Encuentre el intervalo más largo en el
que se debe encontrar \(p^*\) para aproximarse a \(p\) con error
relativo máximo de
\(10^{-4}\)}{3. Encuentre el intervalo más largo en el que se debe encontrar p\^{}* para aproximarse a p con error relativo máximo de 10\^{}\{-4\}}}\label{encuentre-el-intervalo-muxe1s-largo-en-el-que-se-debe-encontrar-p-para-aproximarse-a-p-con-error-relativo-muxe1ximo-de-10-4}

\paragraph{\texorpdfstring{a.
\(\pi\)}{a. \textbackslash pi}}\label{a.-pi}

\(10^{-4} \geq \left| \frac{p - p^*}{p} \right|\)

Despejando \(p^*\):\\
\(p^* = \pi + 10^{-4} \times \pi\)\\
\(p^* = 3.141305032192064\)

\paragraph{\texorpdfstring{b. \(e\)}{b. e}}\label{b.-e}

\(10^{-4} \geq \left| \frac{p - p^*}{p} \right|\)

Despejando \(p^*\):\\
\(p^* = e + 10^{-4} \times e\)\\
\(p^* = 2.718032962324848\)

\paragraph{\texorpdfstring{c.~\(\sqrt{2}\)}{c.~\textbackslash sqrt\{2\}}}\label{c.-sqrt2}

\(10^{-4} \geq \left| \frac{p - p^*}{p} \right|\)

Despejando \(p^*\):\\
\(p^* = \sqrt{2} + 10^{-4} \times \sqrt{2}\)\\
\(p^* = 1.4140840872544698\)

\paragraph{\texorpdfstring{d.~\(\sqrt[3]{7}\)}{d.~\textbackslash sqrt{[}3{]}\{7\}}}\label{d.-sqrt37}

\(10^{-4} \geq \left| \frac{p - p^*}{p} \right|\)

Despejando \(p^*\):\\
\(p^* = \sqrt[3]{7} + 10^{-4} \times \sqrt[3]{7}\)\\
\(p^* = 1.9127560486919348\)

\begin{center}\rule{0.5\linewidth}{0.5pt}\end{center}

\subsubsection{4. Use la aritmética de redondeo de tres dígitos para
realizar lo
siguiente}\label{use-la-aritmuxe9tica-de-redondeo-de-tres-duxedgitos-para-realizar-lo-siguiente}

\paragraph{\texorpdfstring{a.
\(\frac{\frac{13}{14} - \frac{5}{7}}{2e - 5.4}\)}{a. \textbackslash frac\{\textbackslash frac\{13\}\{14\} - \textbackslash frac\{5\}\{7\}\}\{2e - 5.4\}}}\label{a.-fracfrac1314---frac572e---5.4}

\(p = \frac{\frac{13}{14} - \frac{5}{7}}{2e - 5.4}\), \(p^* = 5.860\)

\textbf{Error Absoluto} =
\(\left| \frac{\frac{13}{14} - \frac{5}{7}}{2e - 5.4} - 5.860 \right|\)\\
\(= 3.796 \times 10^{-4}\)

\textbf{Error Relativo} =
\(\frac{\left| \frac{\frac{13}{14} - \frac{5}{7}}{2e - 5.4} - 5.860 \right|}{\left| \frac{\frac{13}{14} - \frac{5}{7}}{2e - 5.4} \right|}\)\\
\(= 0.647 \times 10^{-4}\)

\paragraph{\texorpdfstring{b.
\(-10\pi + 6e - \frac{3}{61}\)}{b. -10\textbackslash pi + 6e - \textbackslash frac\{3\}\{61\}}}\label{b.--10pi-6e---frac361}

\(p = -10\pi + 6e - \frac{3}{61}\), \(p^* = 5.860\)

\textbf{Error Absoluto} =
\(\left| -10\pi + 6e - \frac{3}{61} - 5.860 \right|\)\\
\(= 4.159 \times 10^{-4}\)

\textbf{Error Relativo} =
\(\frac{\left| -10\pi + 6e - \frac{3}{61} - 5.860 \right|}{\left| -10\pi + 6e - \frac{3}{61} \right|}\)\\
\(= 0.274 \times 10^{-4}\)

\paragraph{\texorpdfstring{c.~\(\left( \frac{2}{9} \right) \times \left( \frac{9}{11} \right)\)}{c.~\textbackslash left( \textbackslash frac\{2\}\{9\} \textbackslash right) \textbackslash times \textbackslash left( \textbackslash frac\{9\}\{11\} \textbackslash right)}}\label{c.-left-frac29-right-times-left-frac911-right}

\(p = \left( \frac{2}{9} \right) \times \left( \frac{9}{11} \right)\),
\(p^* = 0.18\)

\textbf{Error Absoluto} =
\(\left| \left( \frac{2}{9} \right) \times \left( \frac{9}{11} \right) - 0.18 \right|\)\\
\(= 1.8182 \times 10^{-11}\)

\textbf{Error Relativo} =
\(\frac{\left| \left( \frac{2}{9} \right) \times \left( \frac{9}{11} \right) - 0.18 \right|}{\left| \left( \frac{2}{9} \right) \times \left( \frac{9}{11} \right) \right|}\)\\
\(= 10^{-10}\)

\paragraph{\texorpdfstring{d.~\(\frac{\sqrt{13} + \sqrt{11}}{\sqrt{13} - \sqrt{11}}\)}{d.~\textbackslash frac\{\textbackslash sqrt\{13\} + \textbackslash sqrt\{11\}\}\{\textbackslash sqrt\{13\} - \textbackslash sqrt\{11\}\}}}\label{d.-fracsqrt13-sqrt11sqrt13---sqrt11}

\(p = \frac{\sqrt{13} + \sqrt{11}}{\sqrt{13} - \sqrt{11}}\),
\(p^* = 23.958\)

\textbf{Error Absoluto} =
\(\left| \frac{\sqrt{13} + \sqrt{11}}{\sqrt{13} - \sqrt{11}} - 23.958 \right|\)\\
\(= 2.607 \times 10^{-4}\)

\textbf{Error Relativo} =
\(\frac{\left| \frac{\sqrt{13} + \sqrt{11}}{\sqrt{13} - \sqrt{11}} - 23.958 \right|}{\left| \frac{\sqrt{13} + \sqrt{11}}{\sqrt{13} - \sqrt{11}} \right|}\)\\
\(= 0.109 \times 10^{-4}\)

\begin{center}\rule{0.5\linewidth}{0.5pt}\end{center}

\subsubsection{\texorpdfstring{5. Aproximaciones de \(\pi\) mediante
polinomio de
Maclaurin}{5. Aproximaciones de \textbackslash pi mediante polinomio de Maclaurin}}\label{aproximaciones-de-pi-mediante-polinomio-de-maclaurin}

Los primeros tres términos diferentes a cero de la serie de Maclaurin
para la función arcotangente son:\\
\$ x - \left( \frac{1}{3} \right)x\^{}3 + \left( \frac{1}{5}
\right)x\^{}5 \$

\paragraph{\texorpdfstring{a.
\(4\arctan\left( \frac{1}{2} \right) + \arctan\left( \frac{1}{3} \right)\)}{a. 4\textbackslash arctan\textbackslash left( \textbackslash frac\{1\}\{2\} \textbackslash right) + \textbackslash arctan\textbackslash left( \textbackslash frac\{1\}\{3\} \textbackslash right)}}\label{a.-4arctanleft-frac12-right-arctanleft-frac13-right}

\$ p\^{}* = 4
\left[ \frac{1}{2} - \left( \frac{1}{3} \right)\left( \frac{1}{2} \right)^3 + \left( \frac{1}{5} \right)\left( \frac{1}{2} \right)^5 \right] \$\\
\$ = 3.1455761316872426 \$

\textbf{Error Absoluto} = \(|\pi - 3.1455761316872426|\)\\
\$ = 0.003983478097449478 \$

\textbf{Error Relativo} = \(\frac{|\pi - 3.1455761316872426|}{\pi}\)\\
\$ = 0.0012679804598147663 \$

\paragraph{\texorpdfstring{b.
\(16\arctan\left( \frac{1}{5} \right) - 4\arctan\left( \frac{1}{239} \right)\)}{b. 16\textbackslash arctan\textbackslash left( \textbackslash frac\{1\}\{5\} \textbackslash right) - 4\textbackslash arctan\textbackslash left( \textbackslash frac\{1\}\{239\} \textbackslash right)}}\label{b.-16arctanleft-frac15-right---4arctanleft-frac1239-right}

\$ p\^{}* = 16
\left[ \frac{1}{5} - \left( \frac{1}{3} \right)\left( \frac{1}{5} \right)^3 - 4 \cdot \frac{1}{239} \left( \frac{1}{3} \right)\left( \frac{1}{239} \right)^3 \right] \$\\
\$ = 3.1750936373416323 \$

\textbf{Error Absoluto} = \(|\pi - 3.1750936373416323|\)\\
\$ = 0.03350098375 \$

\textbf{Error Relativo} = \(\frac{|\pi - 3.1750936373416323|}{\pi}\)\\
\$ = 0.01066369432 \$

\begin{center}\rule{0.5\linewidth}{0.5pt}\end{center}

\subsubsection{\texorpdfstring{6. Aproximaciones de \(e\) mediante
serie}{6. Aproximaciones de e mediante serie}}\label{aproximaciones-de-e-mediante-serie}

El número \(e\) se puede definir por medio de
\(e = \sum_{n=0}^{\infty} \left( \frac{1}{n!} \right)\)

\paragraph{\texorpdfstring{a.
\(\sum_{n=0}^{5} \left( \frac{1}{n!} \right)\)}{a. \textbackslash sum\_\{n=0\}\^{}\{5\} \textbackslash left( \textbackslash frac\{1\}\{n!\} \textbackslash right)}}\label{a.-sum_n05-left-frac1n-right}

\$ p\^{}* = \sum\_\{n=0\}\^{}\{5\} \left( \frac{1}{n!} \right) \$\\
\$ = 2.716666666666663 \$

\textbf{Error Absoluto} = \(|e - 2.716666666666663|\)\\
\$ = 0.0016151617923787498 \$

\textbf{Error Relativo} = \(\frac{|e - 2.716666666666663|}{e}\)\\
\$ = 0.0005941848175817597 \$

\paragraph{\texorpdfstring{b.
\(\sum_{n=0}^{10} \left( \frac{1}{n!} \right)\)}{b. \textbackslash sum\_\{n=0\}\^{}\{10\} \textbackslash left( \textbackslash frac\{1\}\{n!\} \textbackslash right)}}\label{b.-sum_n010-left-frac1n-right}

\$ p\^{}* = \sum\_\{n=0\}\^{}\{10\} \left( \frac{1}{n!} \right) \$\\
\$ = 2.7182818011463845 \$

\textbf{Error Absoluto} = \(|e - 2.7182818011463845|\)\\
\$ = 2.7312660577649694 \times 10\^{}\{-8\} \$

\textbf{Error Relativo} = \(\frac{|e - 2.7182818011463845|}{e}\)\\
\$ = 1.0047766310211053 \times 10\^{}\{-8\} \$

\begin{center}\rule{0.5\linewidth}{0.5pt}\end{center}

\subsubsection{\texorpdfstring{7. Intersección \(x\) de una línea
recta}{7. Intersección x de una línea recta}}\label{intersecciuxf3n-x-de-una-luxednea-recta}

Fórmulas para encontrar la intersección \(x\):

\[
x = \frac{x_0 y_1 - x_1 y_0}{y_1 - y_0}
\]

\[
x = x_0 - \frac{(x_1 - x_0) y_0}{y_1 - y_0}
\] \#\#\#\# a. Con datos \((x_0, y_0) = (1.31, 3.24)\) y
\((x_1, y_1) = (1.93, 5.76)\)

\textbf{Primera fórmula:}\\
\$ x = \frac{1.31 \times 5.76 - 1.93 \times 3.24}{5.76 - 3.24} \$\\
\$ = 0.513 \$

\textbf{Segunda fórmula:}\\
\$ x = 1.31 - \frac{(1.93 - 1.31) \times 3.24}{5.76 - 3.24} \$\\
\$ = 0.513 \$

\textbf{Respuesta:} Ambos métodos son igualmente buenos porque producen
el mismo resultado correcto. Pero, la segunda fórmula podría
considerarse ligeramente más intuitiva si ya conoces uno de los puntos
\((x_0, y_0)\), además realiza menos multiplicaciones y por este motivo
se la puede considerar más exacta.


    % Add a bibliography block to the postdoc
    
    
    
\end{document}
